%\documentclass[11pt,professionalfonts,hyperref={pdftex,pdfpagemode=none,pdfstartview=FitH}]{beamer}
%\usepackage{times}
\documentclass[11pt,professionalfonts]{beamer}
\usefonttheme{serif}
\usepackage{presentation_packages}
\usepackage{multimedia}
\usepackage{animate} % for animated gif


\definecolor{mygray}{gray}{0.9}
\definecolor{RoyalBlue}{rgb}{0.25,0.41,0.88}
\def\Emph{\textcolor{RoyalBlue}}

\definecolor{tmp}{rgb}{0.804,0.941,1.0}
\setbeamercolor{numerical}{fg=black,bg=tmp}
\setbeamercolor{exact}{fg=black,bg=red}

\mode<presentation> 
{
  \usetheme{Warsaw}
  \usefonttheme{serif}
  \setbeamercovered{transparent}
}

\setbeamertemplate{footline}%{split theme}
{%
  \leavevmode%
  \hbox{\begin{beamercolorbox}[wd=.5\paperwidth,ht=2.5ex,dp=1.125ex,leftskip=.3cm,rightskip=.3cm plus1fill]{author in head/foot}%
    \usebeamerfont{author in head/foot}\insertshorttitle
  \end{beamercolorbox}%
  \begin{beamercolorbox}[wd=.5\paperwidth,ht=2.5ex,dp=1.125ex,leftskip=.3cm,rightskip=.3cm]{title in head/foot}
%    \usebeamerfont{title in head/foot}\mypaper\hfill \insertframenumber/\inserttotalframenumber
    \usebeamerfont{title in head/foot}\hfill \insertframenumber/\inserttotalframenumber
  \end{beamercolorbox}}%
  \vskip0pt%
} \setbeamercolor{box}{fg=black,bg=yellow}


\title[2016 SSPI]{\large\bf  Geometric Adaptive Control of Attitude Dynamics on \(\SO \) with State Inequality Constraints}

\author{\vspace*{-0.3cm}}

%\institute{\footnotesize
%{\normalsize Shankar Kulumani}\vspace*{0.2cm}\\
%  Flight Dynamics and Control Lab \\ 
%  Dept. of Mechanical and Aerospace Engineering\\ 
%  The George Washington University \\
%  Washington, DC\\ \vspace{10pt}
%  June 4, 2014 \\ \vspace{10pt}
%  }
   
\institute{
	\footnotesize
	{\normalsize\bf{Shankar Kulumani and Christopher Poole}}
	\vspace*{0.2cm}
  	\bf{Flight Dynamics \& Control Lab}\\ \vspace*{0.5cm}
 	\begin{figure} %figure%
       	\includegraphics[width=0.75\textwidth]{gw_txh_2cs_pos}
  	\end{figure}
}
\date{}

\begin{document}
%=======================================================%

\setcounter{framenumber}{-1}
\begin{frame} %-----------------------------%
  \titlepage
\end{frame}   %-----------------------------%

\section*{}
\subsection*{Introduction}  

\begin{frame}[t]{Motivation} %-----------------------------%
\begin{itemize}
	\item Autonomous control of space vehicles is a critical
	\begin{itemize}
		\item Avoid extensive planning and interaction by operators
		\item Ability to operate safely with uncertainty 
		\item Independently navigate hazards and handle possible failures
	\end{itemize}
\end{itemize}
\visible<2->{
\begin{figure} 
	\centering 	
	\begin{subfigure}[b]{0.5\textwidth} 
		\includegraphics[height=0.35\textheight]{hubble}
	\end{subfigure}~\quad~
	\begin{subfigure}[b]{0.5\textwidth} 
		\includegraphics[height=0.35\textheight]{ors-1}
	\end{subfigure}
\end{figure}
}
\end{frame}   %-----------------------------%

\begin{frame}[t]{Problem} %-----------------------------------%
\begin{itemize}
	\item \Emph{Rigid body attitude control} : reorient vehicle while avoiding pointing at obstacles
	\begin{itemize}
		\item Exclusion zones for payloads e.g infrared telescope
		\item UAVs manuevering in congested locations
		\item Laser/Radio emitters on spacecraft
	\end{itemize}
	\pause
	\item Previous approaches have several issues
	\begin{itemize}
		\item Attitude parameterizations: singularities/ambiguities
		\item Ad-hoc geometric approach: difficult to generalize 
		\item Randomized methods: lack of stability guarantees
		\item Optimization based: expensive to compute and only open-loop control  
	\end{itemize}
\end{itemize}
\end{frame} %-------------------------------------%

\begin{frame}{Objective} %---------------------------------------%

	\begin{block}{Control input}
		Design control input \( u \) that stabilizes system from initial attitude \( R_0 \) to desired attitude \( R_d \) while avoiding obstacles
	\end{block}
	\pause
	\begin{itemize}
		\item Avoid drawbacks of other approaches 
		\begin{itemize}
			\item \Emph{Geometric Control} - analysis is conducted directly on \( \SO \) 
			\item \Emph{Barrier function} - allows for arbitrary amount of constraints
			\item \Emph{Efficient } - real time feedback control
			\item \Emph{Stability} - Lyapunov analysis gives rigourous stability proof
		\end{itemize}
	\end{itemize}
\end{frame}


\begin{frame}{Spacecraft Orientation} %-----------------------------%

\begin{itemize}

	\item \Emph{Attitude Representation}: rotation matrix from body to inertial frame
	 \[\SO =  \{R\in\R^{3\times 3}\,|\, R^TR=I,\;\mathrm{det}[R]=1\} \]
	\item Rigid body attitude dynamics:
	\begin{gather*}
		J\dot\Omega + \Omega\times J\Omega = u+W(R,\Omega)\Delta \quad \dot R = R\hat\Omega
	\end{gather*}

	\item Obstacles defined by units vectors in \( \R^3 \) 
		\begin{itemize}
			\item Body fixed sensor: \( r \in \S^2\)
			\item Inertially fixed hazard: \( v \in \S^2 \)
		\end{itemize} 
	\item Hard Cone constraint: \( r^T R^T v \leq \cos \theta \)
	
\end{itemize}
\end{frame}   %-----------------------------%

\section*{}
\subsection*{Attitude Control}  

\begin{frame}{Configuration Error Function} %-----------------------------%
\only<1>{
\begin{itemize}
	\item Error function quantifies ``distance'' to desired attitude
        \[
        	\Psi(R) = A(R) B(R) 
        \]
	\item Combination of attractive and repulsive terms   
\end{itemize}
\begin{gather*}
	A(R) = \frac{1}{2} \tr{G \left( I - R_d^T R\right)} \\ \\
	B_i(R) = 1 - \frac{1}{\alpha_i} \ln \left( - \frac{ r^T R^T v_i - \cos \theta_i}{1 + \cos \theta_i}\right)
\end{gather*}     
}

\only<2>{
	\begin{itemize}
		\item Attractive well at desired attitude
	\end{itemize}
	\[
	A(R) = \frac{1}{2} \tr{G \left( I - R_d^T R\right)}
	\]
	\begin{figure} 
		\centering
		\includegraphics[height=0.6\textheight]{attract_error}
	\end{figure}
}

\only<3>{
	\begin{itemize}
		\item Define regions of large error around obstacles
	\end{itemize}
	\[
	B_i(R) = 1 - \frac{1}{\alpha_i} \ln \left( - \frac{ r^T R^T v_i - \cos \theta_i}{1 + \cos \theta_i}\right)
	\]
	\begin{figure}
		\centering
		\includegraphics[height=0.6\textheight]{avoid_error}
	\end{figure}

}

\only<4>{
	\begin{itemize}
		\item Control is chosen to follow the slope of the error function towards the minimum at the desired attitude
	\end{itemize}
	\[
		\Psi(R) = A(R) B(R) 
	\]
	\begin{figure}
		\centering
		\includegraphics[height=0.6\textheight]{combined_error}
	\end{figure}
}
\end{frame}   %-----------------------------%


\begin{frame}{Adaptive Attitude Control} %-----------------------------%
\begin{itemize}
	\item Control input: function of \( \Psi \) to achieve stability properties
	\item Adaptive term handles constant unknown disturbances 
		\begin{itemize}
			\item Stuck actuator or gravity gradient moment
		\end{itemize}
\end{itemize}
\pause
	\begin{block}{Adaptive Attitude Controller}
		Zero equilibrium of error vectors are Lyapunov stable, furthermore \( e_R , e_\Omega \to 0 \) as \( t \to \infty \)
		\begin{align*}
			u &= - k_R e_R - k_{\Omega} e_{\Omega} + \Omega \times J \Omega - W \bar{\Delta} \\
			\dot{\bar{\Delta}} &= k_\Delta W^T \parenth{ e_\Omega + c e_R }
		\end{align*}
	\end{block}
	
\end{frame}   %-----------------------------%

\begin{frame}{Numerical Simulation} %-----------------------------%

\begin{itemize}
	\item Simulate a S/C completing a yaw rotation
	\item Single obstacle in the path of sensor
\end{itemize}

\animategraphics[autoplay,loop,width=0.5\textwidth]{8}{./animation/single_noavoid/single_noavoid-}{0}{99}~
\animategraphics[autoplay,loop,width=0.5\textwidth]{8}{./animation/single_avoid/single_avoid-}{0}{99}
\end{frame}%-----------------------------%

\begin{frame}{Multiple obstacles}%-------------------------------------%

\begin{itemize}
	\item Easily handle multiple arbitrary constraints 
	\begin{align*}
		\Psi = A(R) \bracket{1 + \sum_i C_i(R)} \quad C_i = B(R) - 1
	\end{align*}
\end{itemize}
\begin{figure}
	\centering
	\animategraphics[autoplay,loop,width=0.5\textwidth]{8}{./animation/multiple_avoid/multiple_avoid-}{0}{99}
\end{figure}

\end{frame}%---------------------------------------%

\begin{frame}{Hexrotor Design} %--------------------%
	\begin{itemize}
		\item Three pairs of counter-rotating propellers
		\item Custom built carbon fiber frame
		\item Multi-threaded C on Linux ORDROID XU3
		\item Onboard IMU and Vicon motion capture system
	\end{itemize}
	
	\begin{figure} 
		\centering 
		\begin{subfigure}[htbp]{0.5\textwidth} 
			\includegraphics[width=\textwidth]{hexrotor} 
		\end{subfigure}~ %add desired spacing between images, e. g. ~, \quad, \qquad, \hfill etc. %(or a blank line to force the subfigure onto a new line) 
		\begin{subfigure}[htbp]{0.5\textwidth} 
			\includegraphics[width=\textwidth]{vn-100-rugged} 
		\end{subfigure}
	\end{figure}
\end{frame} %-----------------------%

\begin{frame}{Hexrotor Experiment} %-----------------------------%
\begin{itemize}
	\item Attached to spherical joint to allow only attitude dynamics
\end{itemize}
\begin{figure}
\centering
\movie[externalviewer,height=0.7\textheight]{\includegraphics[height=0.7\textheight]{hexrotor}}{experiment.mp4} 
\end{figure}
\end{frame}   %-----------------------------%

\begin{frame}{Conclusions} %-----------------------------%

	\begin{itemize}
		\item Constrained geometric adaptive controller on \( \SO \)
		\begin{itemize}
			\item Completely avoids singularities and ambiguities 
			\item Geometrically exact and simple attitude controller
			\item Automatically satisfies multiple constraints 
		\end{itemize}
		\pause
		\item Obstacle avoidance computed onboard and in real-time
		\begin{itemize}
			\item Ideal for aggressive global manuevers
			\item Computationally efficient and ideal for embedded systems 
			\item Rigourous stability proof 
		\end{itemize}
	\end{itemize}

\end{frame}   %-----------------------------%

\section*{}
\subsection*{Backup}

\begin{frame}[noframenumbering,label=parameterizations]{Attitude Parameterizations}
	\begin{itemize}
		\item Euler Angles
		\begin{itemize}
			\item Minimal representation used for small attitude changes.
			\item Singularities exist for large angle slews: requires switching between 24 sequences
			\item Complicated trigonometric functions
		\end{itemize}
		\pause
		\item Quaternion 
		\begin{itemize}
			\item No singularities
			\item Two anti-podal quaternions for the same attitude
			\item Unwinding behavior 
		\end{itemize}
		\pause
		\item Geometric control
		\begin{itemize}
			\item Globally and uniquely characterize attitude: \( R \in \SO \)
			\item Singular representation of attitude 
			\item Geometrically exact
		\end{itemize}
	\end{itemize}
	
\end{frame}

\begin{frame}[noframenumbering,label=proof]{Lyapunov Analysis}
\begin{itemize}
	\item Positive definite Lyapunov function
	\begin{gather*}
		\mathcal{V} = \frac{1}{2} e_\Omega \cdot J e_\Omega + k_R \Psi + c J e_\Omega \cdot e_R + \frac{1}{2 k_\Delta} e_\Delta \cdot e_\Delta . \label{eqn:v_adapt}
	\end{gather*}
	\item Negative semi-definite \( \dot{\mathcal{V}} \) with \(\zeta=[\|e_R\|,\|e_\Omega\|]\in\R^2\)
	\begin{gather*}
		\dot{\mathcal{V}} \leq -\zeta^T M \zeta .
	\end{gather*}
	\item LaSalle's invariance theorem implies that \(\lim_{t\to\infty} \zeta=0\).
\end{itemize}
\end{frame}



\end{document}

